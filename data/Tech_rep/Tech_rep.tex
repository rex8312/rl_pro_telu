%%%%%%%%%%%%%%%%%%%%%%% file template.tex %%%%%%%%%%%%%%%%%%%%%%%%%
%
% This is a general template file for the LaTeX package SVJour3
% for Springer journals.          Springer Heidelberg 2010/09/16
%
% Copy it to a new file with a new name and use it as the basis
% for your article. Delete % signs as needed.
%
% This template includes a few options for different layouts and
% content for various journals. Please consult a previous issue of
% your journal as needed.
%
%%%%%%%%%%%%%%%%%%%%%%%%%%%%%%%%%%%%%%%%%%%%%%%%%%%%%%%%%%%%%%%%%%%
%
% First comes an example EPS file -- just ignore it and
% proceed on the \documentclass line
% your LaTeX will extract the file if required
\begin{filecontents*}{example.eps}
%!PS-Adobe-3.0 EPSF-3.0
%%BoundingBox: 19 19 221 221
%%CreationDate: Mon Sep 29 1997
%%Creator: programmed by hand (JK)
%%EndComments
gsave
newpath
  20 20 moveto
  20 220 lineto
  220 220 lineto
  220 20 lineto
closepath
2 setlinewidth
gsave
  .4 setgray fill
grestore
stroke
grestore
\end{filecontents*}
%
\RequirePackage{fix-cm}
%
%\documentclass{svjour3}                     % onecolumn (standard format)
%\documentclass[smallcondensed]{svjour3}     % onecolumn (ditto)
\documentclass[smallextended]{svjour3}       % onecolumn (second format)
%\documentclass[twocolumn]{svjour3}          % twocolumn
%
\smartqed  % flush right qed marks, e.g. at end of proof
%
\usepackage{graphicx}
\usepackage{amsmath}
\usepackage{amssymb}
%\usepackage{algorithm}
%
% \usepackage{mathptmx}      % use Times fonts if available on your TeX system
%
% insert here the call for the packages your document requires
%\usepackage{latexsym}
% etc.
%
% please place your own definitions here and don't use \def but
% \newcommand{}{}
%
% Insert the name of "your journal" with
% \journalname{myjournal}
%
\begin{document}

\title{Technical Report: Implementation of DDPG and MPO on virtual and existing environments%\thanks{Grants or other notes
%about the article that should go on the front page should be
%placed here. General acknowledgments should be placed at the end of the article.}
}
\subtitle{Magnetic Levitation, Furuta Pendulum and Ball Balancer}

%\titlerunning{Short form of title}        % if too long for running head

\author{Luca \textit{Hier k\"onnte ihre Werbung stehen} Dziarski         \and
        Theo Gruner %etc.
}

%\authorrunning{Short form of author list} % if too long for running head

\institute{Jan Peters \at
              first address \\
              Tel.: +123-45-678910\\
              Fax: +123-45-678910\\
              \email{fauthor@example.com}           %  \\
%             \emph{Present address:} of F. Author  %  if needed
           \and
           S. Author \at
              second address
}

\date{Received: date / Accepted: date}
% The correct dates will be entered by the editor


\maketitle

\begin{abstract}
The model-free algorithms \textit{Deep Deterministic Policy Gradient} (DDPG) and \textit{Maximum A Posteriori Policy Optimization} (MPO) have been implemented and tested on a variety of tasks. First they were tested in an virtual environment after which they were evaluated using real robots. This paper contains the implementation details, specific variations from the original algorithms and a detailed evaluation section. Evaluations have been carried out with focus on the hyper-parameter settings for each individual algorithm and experiment. This paper supplements the reports of many other research groups at the TU Darmstadt to found an extensive summary of RL algorithms on a variety of popular RL environments.
\keywords{Deep Reinforcement Learning \and DDPG \and MPO}
% \PACS{PACS code1 \and PACS code2 \and more}
% \subclass{MSC code1 \and MSC code2 \and more}
\end{abstract}

\section{Introduction}

\begin{itemize}
	\item General:
	\begin{itemize}
		\item Robot learning
		\item curse of high dimensionality
		\item difficulties of 
	\end{itemize}
	\item continuous action and state-space
	\begin{itemize}
		\item
	\end{itemize}
	\item model free algorithms
	\item advantage of DDPG \cite{Lillicrap2015} and MPO \cite{Abdolmaleki2018a} briefly
\end{itemize}
\section{Explanation of the used Algorithms}\label{sec_algorihtms}
The reviewed algorithms are model-free and off-policy.  
\subsection{Deep Deterministic Policy Gradient}
DDPG \cite{Lillicrap2015} is a policy search algorithm which optimizes the policy based on gradient descent. It is assumed that the gradient of the policy gradient and the gradient of the objective function $\mathcal{J}(\theta)$ w.r.t. the policy parameters $\theta$ are in the same direction which results in following policy update.
\begin{equation}
	\theta_{t+1} = \theta_{t} + \alpha \nabla_{\theta}\mathcal{J}(\theta)
\end{equation} 
The evaluation of $\nabla_{\theta}\mathcal{J}(\theta)$ is based on the Deterministic Policy Gradient \cite{Silver2014}.
\begin{equation}
	\nabla_{\theta}\mathcal{J}(\theta) = \mathbb{E}_{s\sim\mu^{\pi}(s)}\left[\nabla_{\theta}Q(s,a)|_{a=\pi(s)}\right]
\end{equation}
The Q-function needed for the calculation of the policy gradient is represented by a neural network. The parameterized policy is also approximated by neural networks. In order to fit the parameterized Q-function to the true one, the Q-function is improved by minimizing the TD-difference between the expected Q-function $Q(s_{t}, a_{t})$ in state $s_{t}$ and the reward $r(s_{t},a_{t}) + Q(s_{t+1},a_{t+1})$
\begin{equation}
	\min\limits_{\phi}L = \min\limits_{\phi} \mathbb{E}_{s\sim\mu^{\pi}(s)}\left[\left(Q^{\phi}(s_{t}, a_{t}) - r(s_{t},a_{t}) + Q^{\phi^{\prime}}(s_{t+1},a_{t+1})\right)^{2}\right]
\end{equation}
The expectation can be approximated by sampling random tuples of $(s_{t}, a_{t}, r(s_{t}, a_{t}), s_{t+1})$.
In order to be off-policy, a Gaussian noise is added to the deterministic policy. As will be shown in the later sections, choosing an appropriate noise is key for the performance of the algorithm.
\paragraph{Useful Hyperparameter Tuning:}
DDPG lacks the stability of robust Trust-Region algorithms and is heavily dependent on the hyperparameter settings. We focused primarily on the usage of different noises and on the batch size. The default noise used in the original paper was an Ornstein-Uhlenback (OU) noise, but recent advances favor the Adaptive-Parameter noise \cite{Plappert2017}. The first adds noise to the action while the latter adds noise to the parameters of a perturbed actor.\\
Another important hyperparameter tends to be the mini-batch size. In order to estimate the mean the batch-size should grow for more complex environments.

\subsection{Maximum A Posteriori Policy Optimization}
Big disadvantages of off-policy algorithms, like DDPG, are that convergence is not guaranteed, especially for high dimensional problems. This is due to an information loss \cite{Peters2008a} occurring when the difference between the old and updated policy is too large. On-policy algorithms, like REPS or PPO, cope with this problem by introducing an additional Kullback-Leibler constraint which regularizes the policy update w.r.t the old policy. On-policy algorithms tend to converge more consistently than off-policy algorithm but are instead data inefficient since the sampled data can only be used once.\\
MPO \cite{Abdolmaleki2018a} uses the KL constraint but is still and off-policy algorithm which leads to an algorithm which is supposed to converge fast with high data efficiency.
The optimization is split into three distinct steps. First, the Q-function is updated, then the dual function of the constraint optimization problem is solved which is similar to REPS and in the third step the policy is updated. Except for the hyperparameters, the evaluation of the Q-function seemed to be to be crucial.\\
Concerning MPO, one can choose between a parametric or non-parametric additional q-function which perform differently. Both ways have been implemented in the context of this work.
\begin{itemize}
	\item Briefly explain background of algorithm
	\item Difficulties in implementation
	\item 
\end{itemize}
\section{Environments used for Evaluation}
The algorithms are evaluated on three different environments which all represent a balancing task in which the desired state is an unstable equilibrium. The considered experiments are \textit{Magnetic Levitation}, \textit{Ball Balancer} and \textit{Furuta Pendulum}. All environments are characterized by a state $s$. For each state $s$, the agent can choose an action and receives a reward from the environment. Since the algorithm can be implemented in a continuous environment, state and action space are continuous.
\subsection{Magnetic Levitation}
The experiments are ordered based on their complexity. For Magnetic Levitation, a ball is placed in the middle of a magnetic field which is generated by an electromagnet. By controlling the coil voltage, the direction and strength of the magnetic field can be regulated which applies a force onto the ball. The desired task is to minimize the deflection of the ball from its neutral position while using the least amount of current to fulfill the task. Therefore, the states are sufficiently described by a distance measure in the vertical direction and the current of the coil.
%TODO Explaining the reward function
\subsection{Furuta Pendulum}
The pendulum consists of two connected arms. The first one is attached to an motor at its end and can rotates around the vertical axis. The end of the first and second arm are aligned perpendicularly. The task is to balance the second arm upright by choosing the right momentum. The system is fully described by the angles of the arm and pole as well as the angular velocity of the arm. For numerical reasons, the sines and cosines of the respective angles are stored resulting in 6 observed states.	
\subsection{Ball Balancer}
observation space is 8 
action space is 2
\section{Evaluation}
The following section summarizes the performance of the previously described algorithms (section \ref{sec_algorihtms}) for the three environments. Thereby, the choice of parameters as well as the difference between the evaluation on the virtual and real environment will be of importance. The evaluation procedure in this section is as follows:\\
\textit{First}, the learning procedure as well as the evaluation is carried out in the virtual environment which serves as a baseline. \textit{Secondly}, the learned models will be evaluated on the real robots. \textit{Thirdly}, learning and evaluation will be carried out on the real robot.
The following table represents hyper-parameter settings which are constant for all environments:
\begin{table}
	\centering
	\caption{General evaluation criteria}
	\begin{tabular}{|c|c|c|}
		\hline
		&DDPG&MPO\\
		\hline
		episode length&$300$&$3000$\\
		\hline
		Actor net&$400-300$&$100 - 100$\\
		\hline
		Critic net&$400-300$&$200-200$\\
		\hline
		mini-batch size&$64$&$64$\\
		\hline
		additional actions&-&$64$\\
		\hline
		$\gamma$&$0.99$&$0.99$\\
		\hline
		learning rate&$0.0005$&$0.0005$\\
		\hline
		$\tau$&$0.001$&-\\
		\hline
		$\varepsilon$&-&$0.1$\\
		\hline
		$\varepsilon_\mu$&-&$5\cdot10^{-4}$\\
		\hline
		$\varepsilon_{\Sigma}$&-&$10^{-5}$\\
		\hline
	\end{tabular}
	\label{table_general_settings}
\end{table}
\subsection{Magnetic Levitation}
\subsection{Furuta Pendulum}
\subsection{Ball Balancer}
\subsection{Evaluation of the algorithms trained in simulation}
\textit{Training results as well as evaluation results on virtual environment}
\subsection{Performance of the learned algorithms on the real robots}
\textit{Using the learned model from the virtual environment on the real robots}
\subsection{Training and Evaluation on the real robots}
\textit{Learning and evaluation on real robot}
\section{Conclusion}


\begin{acknowledgements}
I am tremendously grateful for my always loving mother, Renate the \textit{GRANATE}. Without her I wouldn't be where I am now (pun intended). Special thanks to the best imaginable flat-sharing community, \textit{WG DA ihr Hurn}, who always believed in me and pushed me to my absolute limits.\qquad \textit{$\sim$ Luca Dziarski}
\end{acknowledgements}

% BibTeX users please use one of
%\bibliographystyle{spbasic}      % basic style, author-year citations
\bibliographystyle{spmpsci}      % mathematics and physical sciences
%\bibliographystyle{spphys}       % APS-like style for physics
\bibliography{Reinforcement-Learning-Project-Technical-Report}   % name your BibTeX data base

% Non-BibTeX users please use
%\begin{thebibliography}{}
%
% and use \bibitem to create references. Consult the Instructions
% for authors for reference list style.
%
%\bibitem{RefJ}
% Format for Journal Reference
Author, Article title, Journal, Volume, page numbers (year)
% Format for books
%\bibitem{RefB}
%Author, Book title, page numbers. Publisher, place (year)
% etc
%\end{thebibliography}

\end{document}
% end of file template.tex

